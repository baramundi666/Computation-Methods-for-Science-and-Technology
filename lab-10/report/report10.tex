\documentclass{article}

\usepackage{hyperref}
\usepackage[T1]{fontenc}
\usepackage{graphicx}
\usepackage{float}
\usepackage[utf8]{inputenc}
\usepackage{amsmath}
\usepackage{amsfonts}


\title{%
Laboratorium 10\\
  \huge Równania różniczkowe - spectral bias}
\author{Mateusz Król}
\date{06/06/2024 r.}

\begin{document}
\maketitle

 
\section*{Zadanie 1.}
\textbf{Dane jest równanie różniczkowe zwyczajne
$$ \frac{du(x)}{dx} = cos(\omega x) \text{ dla } x \in \Omega \text{,}  $$
gdzie: \\
$x, \omega, u \in \mathbb{R}$,\\
x to położenie,\\
$\Omega$ to dziedzina, na której rozwiązujemy równanie, \\
$\Omega = \{x \text{  }| -2\pi \le x \le 2\pi\}$, \\
$u(\cdot)$ to funkcja, której postaci szukamy. \\\\
Warunek początkowy zdefiniowany jest następująco:
$$ u(0) = 0 \text{.}$$
Analityczna postać rozwiązania równania z warunkiem początkowym
jest następująca:
$$ u(x) = \frac{1}{\omega} \sin(\omega x) $$
Rozwiąż powyższe zagadnienie początkowe. Do rozwiązania użyj sieci neuronowych
PINN (ang. \textit{Physics-informed Neural Network} ) 
Można wykorzystać szablon w pytorch-u lub bibliotekę DeepXDE.
}\\\\


\section*{Zadanie 2.}
\textbf{Dane jest równanie różniczkowe zwyczajne
$$y'=-5y$$
z warunkiem początkowym $y(0) = 1$. Równanie rozwiązujemy numerycznie z
krokiem $h = 0.5$.
}\\\\

Poniższa tabela przedstawia wartości czynników amplifikacji dla zadanego
równania różniczkowego, dla każdej z omówionych metod.\\
Wartości te świadczą o numerycznej stabilności tych metod.

\begin{center}
  \begin{tabular}{c c} 
   Method & Amplification factor\\
   Explicit Euler & $-1.5$\\
   Implicit Euler & $\approx 0.286$\\
   Trapezoidal  & $\approx -0.111$\\
   Modified Euler & $1.625$\\
   RK4 & $\approx 0.648$
  \end{tabular}
\end{center}

Metody stabilne numerycznie dla zadanego równania to te, dla których
wartość bezwzględna z czynnika amplifikacji jest mniejsza od 1.\\
Są to metody: Implicit Euler, Trapezoidal, RK4.\\\\

Jawna metoda \textit{Euler}'a jest niestabilna numerycznie dla zadanego
parametru $h=0.5$.\\
Wzór iteracyjny:
$$y_{n+1} = y_n + h_n\lambda y_n$$
Wartość przybliżonego rozwiązania tą metodą dla $t=0.5$ wynosi
$y=-1.5$.\\


Niejawna metoda \textit{Euler}'a jest stabilna numerycznie dla zadanego
parametru $h=0.5$.\\
Wzór iteracyjny:
$$y_{n+1} = \frac{y_n}{1 - h_n\lambda}$$
Wartość przybliżonego rozwiązania tą metodą dla $t=0.5$ wynosi
$y\approx 0.286$.\\



\section*{Zadanie 3.}
\textbf{Model \textit{Kermack}’a-\textit{McKendrick}’a przebiegu epidemii w populacji
opisany jest układem równań różniczkowych:
$$S'=-\beta I S$$
$$I'=\beta I S - \gamma I$$
$$R'=\gamma I$$
, gdzie: \\
S reprezentuje liczbę osób zdrowych, podatnych na zainfekowanie,\\
I reprezentuje liczbę osób zainfekowanych i roznoszących infekcję,\\
R reprezentuje liczbę osób ozdrowiałych.\\\\
Parametr $\beta$ reprezentuje współczynnik zakaźności (ang. \textit{transmission rate}).\\
Parametr $\gamma$ reprezentuje współczynnik wyzdrowień (ang. \textit{recovery rate}).\\
Wartość $\frac{1}{\gamma}$ reprezentuje średni czas choroby. \\\\
Założenia modelu:
}

$$L_1(\theta) = \sum_{t=0}^{T} (I_t - \hat{I_t})^2$$
$$L_2(\theta) = -\sum_{t=0}^{T} I_t\cdot \ln(\hat{I_t}) + 
\sum_{t=0}^{T} \hat{I_t}$$

Tabela przedstawiająca wyznaczone wartości współczynników $\beta$ i $\gamma$
korzystając z metody \textit{Rungego Kutty} dla różnych
funkcji kosztu (przyjmując $N=500$):
\begin{center}
  \begin{tabular}{c c c c} 
   Cost function & $\beta$ & $\gamma$ & $R_0 = \frac{\beta}{\gamma}$\\
   $L_1$ & $\approx1.568$ & $\approx0.283$ & $\approx5.547$\\
   $L_2$ & $\approx1.583$ & $\approx0.330$ & $\approx4.811$
  \end{tabular}
\end{center}


\end{document}