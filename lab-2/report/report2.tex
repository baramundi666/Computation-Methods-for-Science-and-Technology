\documentclass{article}

\usepackage{hyperref}
\usepackage[T1]{fontenc}
\usepackage{graphicx}
\usepackage[utf8]{inputenc}


\title{%
Laboratorium 2\\
  \huge Metoda najmniejszych kwadratów}
\author{Mateusz Król}
\date{12/03/2024 r.}

\begin{document}
\maketitle


\section*{Zadanie 1.}
\textbf{Zastosuj metodę najmniejszych kwadratów do predykcji, czy nowotwór jest złośliwy (ang. malignant) czy łagodny (ang. benign).} 
\\\\
abc
\subsection*{Wnioski}
\null\quad bruh



\section*{Źródła}
\begin{itemize}
    \item \url{https://en.wikipedia.org/wiki/Single-precision_floating-point_format}
    \item \url{https://en.wikipedia.org/wiki/Double-precision_floating-point_format}
\end{itemize}



\end{document}