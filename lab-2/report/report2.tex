\documentclass{article}

\usepackage{hyperref}
\usepackage[T1]{fontenc}
\usepackage{graphicx}
\usepackage[utf8]{inputenc}


\title{%
Laboratorium 2\\
  \huge Metoda najmniejszych kwadratów}
\author{Mateusz Król}
\date{12/03/2024 r.}

\begin{document}
\maketitle


\section*{Zadanie 1.}
\textbf{Zastosuj metodę najmniejszych kwadratów do predykcji, czy nowotwór 
jest złośliwy (ang. malignant) czy łagodny (ang. benign).
Nowotwory złośliwe i łagodne mają różne charakterystyki wzrostu. Istotne cechy to m. in. promień i tekstura. Charakterystyki te wyznaczane są poprzez
diagnostykę obrazową i biopsje.} 
\\\\
Przykładowe dane:
\begin{figure}[ht!]
  \includegraphics[width=\linewidth]{figures/hist.png}
\end{figure}
\begin{figure}[ht!]
  \includegraphics[width=\linewidth]{figures/plot.png}
\end{figure}
\\
Korzystając z $A_{lin}$ - liniowej reprezentacji oraz $A_{quad}$ - 
kwadratowej reprezentacji danych z pierwszego zbioru dostarczonego do zadania, jesteśmy w 
stanie wyznaczyć wektor wag $w$:
$$ A^T \cdot A \cdot w = A^T \cdot b$$
W tym celu wykorzystuję funkcję \textit{scipy.linalg.solve}, służącą do rozwiązywania 
układów równań liniowych, z biblioteki \textit{SciPy}. \\\\
Współczynniki uwarunkowania macierzy liczę korzystając z funkcji \textit{numpy.linalg.cond} 
z biblioteki \textit{NumPy}.
Wartości wyniosły:
\begin{itemize}
  \item dla reprezentacji liniowej - $cond(A^T\cdot A)\approx 1.81\cdot 10^{12}$
  \item dla reprezentacji kwadratowej - $cond(A^T\cdot A)\approx 9.06\cdot 10^{17}$
\end{itemize}
Następnie wyznaczając wektor $p$ na podstawie danych z drugiego zbioru dostarczonego do
zadania i porównując go z wartościami prawdziwymi, jesteśmy w stanie oszacować dokładność
metody dla różnych reprezentacji macierzowych danych:
\begin{center}
  \begin{tabular}{  |c|c|c| } 
   \hline
   Matrix representation & false-positives & false-negatives\\
   \hline
   linear & 6 & 2\\
   \hline
   quadratic & 15 & 5\\
   \hline
  \end{tabular}
\end{center}
Dokładność reprezentacji macierzowych:
\begin{itemize}
  \item linear: $accuracy = \frac{260-6-2}{260} \approx 96.9 \%$
  \item quadratic: $accuracy = \frac{260-15-5}{260} \approx 93.5 \%$
\end{itemize}
  \subsection*{Wnioski}
\null\quad Reprezentacja liniowa okazała się być dokładniejsza dla dostarczonych
zbiorów danych - zwróciła mniejszą
ilość wyników fałszywych niż reprezentacja kwadratowa. Wynika to zapewne
z faktu, że dostarczyliśmy w reprezentacji kwadratowej mniej danych - 
zaledwie 4 kolumny. \\
Wykorzystanie w metodzie reprezentacji liniowej, zapewniło 
mniejszy współczynnik uwarunkowania. Błąd reprezentacji 
numerycznej zbiorów danych powoduje w reprezentacji kwadratowej
większy błąd wyniku.


\end{document}