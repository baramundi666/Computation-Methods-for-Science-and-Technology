\documentclass{article}

\usepackage{hyperref}
\usepackage[T1]{fontenc}
\usepackage{graphicx}
\usepackage{float}
\usepackage[utf8]{inputenc}
\usepackage{amsmath}
\usepackage{amsfonts}


\title{%
Laboratorium 11\\
  \huge Optymalizacja}
\author{Mateusz Król}
\date{15/06/2024 r.}

\begin{document}
\maketitle

 
\section*{Zadanie 1.}
\textbf{Wyznacz punkty krytyczne każdej z poniższych funkcji.
Scharakteryzuj każdy znaleziony punkt jako minimum, maksimum lub punkt siodłowy.
Dla każdej funkcji zbadaj, czy posiada minimum globalne lub maksimum globalne na zbiorze $\mathbb{R}$.}
$$f_1(x, y) = x^2 - 4xy + y^2$$
$$f_2(x, y) = x^4 - 4xy + y^4$$
$$f_3(x, y) = 2x^3 - 3x^2 -6xy(x-y-1)$$
$$f_4(x, y) = (x-y)^4 + x^2 - y^2 - 2x + 2y +1$$
\\\\
Dla funkcji $f_1$:
$$\frac{\partial f_1}{\partial x} = 2x-4y = 0$$
$$\frac{\partial f_1}{\partial y} = -4x+2y = 0$$
Punkty krytyczne:
$$(x, y) = (0, 0)$$
Macierz Hessego:
$$H = \begin{bmatrix}
  2 & -4 \\
  -4 & 2 
\end{bmatrix}$$
$$det(H) = -12 < 0$$
Punkt $(0, 0)$ jest punktem siodłowym. \\


Dla funkcji $f_2$:
$$\frac{\partial f_2}{\partial x} = 4x^3-4y = 0$$
$$\frac{\partial f_2}{\partial y} = -4x+4y^3 = 0$$
Punkty krytyczne:
$$(x, y) = (0, 0), (1, 1), (-1, -1)$$
Macierz Hessego:
$$H = \begin{bmatrix}
  12x^2 & -4 \\
  -4 & 12y^2
\end{bmatrix}$$
$$det(H(0, 0)) = -16 < 0$$
Punkt $(0, 0)$ jest punktem siodłowym.
$$det(H(1, 1)) = 128 > 0$$
$$\frac{\partial^2 f_1}{\partial x^2}(1, 1) = 12 > 0$$
Punkt $(1, 1)$ jest minimum lokalnym.
$$det(H(-1, -1)) = 128 > 0$$
$$\frac{\partial^2 f_1}{\partial x^2}(-1, -1) = 12 > 0$$
Punkt $(-1, -1)$ jest minimum lokalnym. \\

Dla funkcji $f_3$:
$$\frac{\partial f_3}{\partial x} = 6x^2-6x-12xy+6y^2+6y = 0$$
$$\frac{\partial f_3}{\partial y} = -6x^2+6xy+6x = 0$$
Punkty krytyczne:
$$(x, y) = (0, 0), (t, t-1) \text{, } t\in \mathbb{R}$$
Macierz Hessego:
$$H = \begin{bmatrix}
  12x-6-12y & -12x+12y+6 \\
  -12x+12y+6 & 0 
\end{bmatrix}$$
$$det(H(0, 0)) = -36 < 0$$
Punkt $(0, 0)$ jest punktem siodłowym. \\
$$det(H(t, t-1)) = -36 < 0$$
Punkty $(t, t-1) \text{, } t\in \mathbb{R}$ śa punktami siodłowymi. \\

Dla funkcji $f_4$:
$$\frac{\partial f_4}{\partial x} = 4(x-y)^3 + 2x - 2 = 0$$
$$\frac{\partial f_4}{\partial y} = -4(x-y)^3 -2y + 2 = 0$$
Punkty krytyczne:
$$(x, y) = (1, 1)$$
Macierz Hessego:
$$H = \begin{bmatrix}
  -12(x-y)^2 + 2 & -12(x-y)^2 \\
  -12(x-y)^2 & 12(x-y)^2-2 
\end{bmatrix}$$
$$det(H(1, 1)) = -4 < 0$$
Punkt $(1, 1)$ jest punktem siodłowym.



\section*{Zadanie 2.}
\textbf{Należy wyznaczyć najkrótszą ścieżkę robota pomiędzy dwoma punktami
$x^{(0)}$ i $x^{(n)}$. \\
Problemem są przeszkody usytuowane na trasie robota, których
należy unikać.\\
Zadanie polega na minimalizacji funkcja kosztu, która sprowadza
problem nieliniowej optymalizacji z ograniczeniami do problemu nieograniczonej
optymalizacji.}
\\\\

Algorytm największego spadku z przeszukiwaniem liniowym (ang. \textit{Gradient Descent with Line Search})
iteracyjnie znajduje minimum funkcji celu poprzez poruszanie się w kierunku przeciwnym do
obliczonego na podstawie gradientu funkcji. Do przeszukiwania liniowego wykorzystuję metodę złotego podziału
(ang. \textit{Golden Section Search}), która znajduje optymalny krok $\alpha$ wzdłuż kierunku gradientu.
\\\\
Kroki algorytmu:
\begin{itemize}
  \item punkty początkowe: $x$\\
  \item przedział poszukiwań dla $\alpha$ \\
  \item iteracyjnie: obliczenie gradientu, znalezienie optymalnego $\alpha$ dla funkcji $F(x_{k} - \alpha \cdot grad(F(x_k)))$
  , aktualizacja $x_{k+1} = x_k - \alpha \cdot grad(F(x_k))$ \\
\end{itemize}

\null\quad
Pięcio krotnie wylosowałem początkowe wartości $x$ z rozkładu jednostajnego z przedziału
$(0; 20)$.\\
\null\quad
Poniżej znajdują się wykresy odpowiadające końcowym wartościom $x$, czyli ścieżce 
\textit{wybranej} przez robota: \\

\begin{figure}[H]
  \includegraphics[width=\linewidth]{figures/1.png}
\end{figure}

\begin{figure}[H]
  \includegraphics[width=\linewidth]{figures/2.png}
\end{figure}

\begin{figure}[H]
  \includegraphics[width=\linewidth]{figures/3.png}
\end{figure}

\begin{figure}[H]
  \includegraphics[width=\linewidth]{figures/4.png}
\end{figure}

\begin{figure}[H]
  \includegraphics[width=\linewidth]{figures/5.png}
\end{figure}

\newpage
Poniżej znajduje się wykres wartości funkcji kosztu $F$ w zależności od dotychczasowej
liczby iteracji:

\begin{figure}[H]
  \includegraphics[width=\linewidth]{figures/F.png}
\end{figure}

Wartości funkcji $F$ bardzo szybko maleją, wydają się zbiegać do wartości $\approx 70.78$.



\end{document}