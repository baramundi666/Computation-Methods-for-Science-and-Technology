\documentclass{article}

\usepackage{hyperref}
\usepackage[T1]{fontenc}
\usepackage{graphicx}
\usepackage{float}
\usepackage[utf8]{inputenc}
\usepackage{amsmath}
\usepackage{amsfonts}


\title{%
Laboratorium 11\\
  \huge Optymalizacja}
\author{Mateusz Król}
\date{15/06/2024 r.}

\begin{document}
\maketitle

 
\section*{Zadanie 1.}
\textbf{Wyznacz punkty krytyczne każdej z poniższych funkcji.
Scharakteryzuj każdy znaleziony punkt jako minimum, maksimum lub punkt siodłowy.
Dla każdej funkcji zbadaj, czy posiada minimum globalne lub maksimum globalne na zbiorze $\mathbb{R}$.}
\newpage

\section*{Zadanie 2.}
\textbf{Należy wyznaczyć najkrótszą ścieżkę robota pomiędzy dwoma punktami
$x^{(0)}$ i $x^{(n)}$. \\
Problemem są przeszkody usytuowane na trasie robota, których
należy unikać.\\
Zadanie polega na minimalizacji funkcja kosztu, która sprowadza
problem nieliniowej optymalizacji z ograniczeniami do problemu nieograniczonej
optymalizacji.}
\newpage



\end{document}