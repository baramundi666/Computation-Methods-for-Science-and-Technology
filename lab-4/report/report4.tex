\documentclass{article}

\usepackage{hyperref}
\usepackage[T1]{fontenc}
\usepackage{graphicx}
\usepackage{float}
\usepackage[utf8]{inputenc}


\title{%
Laboratorium 4\\
  \huge Efekt Rungego}
\author{Mateusz Król}
\date{03/04/2024 r.}

\begin{document}
\maketitle


\section*{Zadanie 1.}
\textbf{Wyznacz wielomiany interpolujące funkcje:}
$$f_1(x) = \frac{1}{1 + 25\cdot x^2} \mbox{ na przedziale } [-1; 1], $$
$$f_2(x) = e^{\cos(x)} \mbox{ na przedziale } [0; 2\pi]$$
\\
\textbf{używając:} 
\begin{itemize}
  \item \textbf{wielomianów Lagrange’a z równoodległymi węzłami}
  \item \textbf{kubicznych funkcji sklejanych z równoodległymi węzłami} 
  \item \textbf{wielomianów Lagrange’a z węzłami Czebyszewa:}
\end{itemize} 
\newpage
Porównanie wykresów wielomianów interpolacyjnych funkcji Rungego: \\

\begin{figure}[H]
  \includegraphics[width=\linewidth]{figures/interpolation.png}
\end{figure}
\newpage
Porównanie norm wektorów błędów wielomianów interpolacyjnych funkcji $f_1$ dla
 $n$ z przedziału $[4;50]$: \\

\begin{figure}[H]
  \includegraphics[width=\linewidth]{figures/errors_1.png}
\end{figure}
\newpage
Porównanie norm wektorów błędów wielomianów interpolacyjnych funkcji $f_2$ dla
 $n$ z przedziału $[4;50]$: \\

\begin{figure}[H]
  \includegraphics[width=\linewidth]{figures/errors_2.png}
\end{figure}



\subsection*{Wnioski}
\null\quad Na podstawie wykresów, jako najbardziej dokładną metodę
interpolacji z przetestowanych, można okreslić interpolację wielomianami
\textit{Lagrange}'a z wykorzystaniem węzłów \textit{Chebyshev}'a. \\
\null\quad Najgorszą okazała się interpolacja zwykłymi wielomianami
\textit{Lagrange}'a.

\section*{Źródła}
\begin{itemize}
    \item \url{https://heath.cs.illinois.edu/scicomp/notes/cs450_chapt07.pdf}
\end{itemize}


\end{document}